\documentclass[fleqn,10pt]{SelfArx} % Document font size and equations flushed left


\definecolor{color1}{RGB}{0,0,90} % Color of the article title and sections
\definecolor{color2}{RGB}{0,20,20} % Color of the boxes behind the abstract and headings

\JournalInfo{Supplementary Text} % Journal information ``Journal, Vol. XXI, No. 1, 1-5, 2015''
\Archive{ } % Additional notes (e.g. copyright, DOI, review/research article)

\PaperTitle{Technical details for: A profile-based method for measuring the impact of genetic variation} % Article title

\Authors{Nicole E. Wheeler\textsuperscript{1}*, Lars Barquist\textsuperscript{2}, Fatemeh Ashari Ghomi\textsuperscript{1}, Robert Kingsley\textsuperscript{3}, Paul P. Gardner\textsuperscript{1,4}} % Authors
\affiliation{\textsuperscript{1}\textit{School of Biological Sciences, University of Canterbury, Christchurch, New Zealand.}} % Author affiliation
\affiliation{\textsuperscript{2}\textit{Institute for Molecular Infection Biology, University of Wuerzburg, Wuerzburg, Germany.}} % Author affiliation
\affiliation{\textsuperscript{3}\textit{Institute of Food Research, Norwich Research Park, Norwich, Norfolk, United Kingdom.}}
\affiliation{\textsuperscript{4}\textit{Biomolecular Interaction Centre and the Bio-Protection Research Centre, University of Canterbury, Christchurch, New Zealand.}}
\affiliation{*\textbf{Corresponding author}: nicole.wheeler@pg.canterbury.ac.nz} % Corresponding author

\Keywords{genome variation --- genotype --- phenotype} % Keywords - if you don't want any simply remove all the text between the curly brackets
\newcommand{\keywordname}{Keywords} % Defines the keywords heading name

%----------------------------------------------------------------------------------------
%	ABSTRACT
%----------------------------------------------------------------------------------------

\Abstract{In the following we provide some mathematical justification
  for the Delta bitscore metric that we evaluate in the accompanying
  manuscript.}

\begin{document}

\flushbottom % Makes all text pages the same height
\maketitle % Print the title and abstract box
%\tableofcontents % Print the contents section

\thispagestyle{empty} % Removes page numbering from the first page

%----------------------------------------------------------------------------------------
%	ARTICLE CONTENTS
%----------------------------------------------------------------------------------------

\section*{Introduction} % The \section*{} command stops section numbering

In this document we compare the mathematics motivating profile HMM
\cite{Krogh:1994,BSA1998} based methods for quantifying the likely
phenotypic significance of genetic variation. We focus on methods
that compare a reference ($ref$) sequence to a variant ($var$)
\cite{Clifford:2004,Shihab:2013,Shihab:2013a,Shihab:2014,Liu:2014,Liu:2015}
These methods are the \emph{logR.E-value} approach
\cite{Clifford:2004}, \emph{FATHMM}
\cite{Shihab:2013,Shihab:2013a,Shihab:2014}, HMMvar
\cite{Liu:2014,Liu:2015} and \emph{DBS} (this study).

\section{Methods}



\subsection{Delta bitscore (DBS)}

We define delta bitscore (DBS) as: 

\begin{eqnarray} 
\label{eq:dbs}
DBS &=& \left(x_{ref} - x_{var}\right)
\end{eqnarray}

The bitscore ($x$) for an HMM is defined as a $\log$ of probability
ratios \cite{BSA1998}:

\begin{eqnarray} 
\label{eq:bs}
x &=& \log_2\left(\frac{P(seq|M)}{P(seq|N)} \right)
\end{eqnarray}

Where $M$ is a profile model derived from a sequence alignment. $M$
can generate and score sequences based upon how likely they are to
have been produced by the same process as those in the sequence
alignment.  $N$ is a null model, that generates and scores sequences
based upon how likely they are to have produced by a random process.

Therefore, $DBS$ can be re-written as:

\begin{eqnarray} 
\label{eq:dbs}
DBS(seq_{ref}, seq_{var}) &=&      \log_2\left(\frac{P(seq_{ref}|M)}{P(seq_{ref}|N)}\right) \nonumber\\ 
\label{eq:dbs1}
                                & - & \log_2\left(\frac{P(seq_{var}|M)}{P(seq_{var}|N)}\right)\\
\label{eq:dbs2}
                               &\approx& \log_2\left(\frac{P(seq_{ref}|M)}{P(seq_{var}|M)} \right)
\end{eqnarray}

If we make the simplifying assumption that the null models for
$P(seq_{ref}|N)$ and $P(seq_{var}|N)$ are approximately equal
(i.e. equal length and amino acid composition).  Therefore, the first
term of Equation~\ref{eq:fathmm1} and Equation~\ref{eq:dbs2} are, in
most situations, equivalent. 



\subsection{logR.E-value}

Clifford \emph{et al.} (2004) suggest using the following measure to
estimate the significance of a genetic variant:

\begin{equation} 
\label{eq:logre}
\log R.E = \log_{10}\left(\frac{E-value_{var}}{E-value_{ref}}\right)
\end{equation}

Where $E-value_{var}$ and $E-value_{ref}$ correspond to the
expectation value derived from HMMER matches (to the same model) for a
variant ($var$) and canonical ($can$) protein sequence. 

$E-values$ are generally estimated by fitting an exponential
distribution to an empirical (usually simulated)
distribution. I.e. 

\begin{equation} 
\label{eq:eval}
E-value = \kappa MN e^{\lambda x}
\end{equation}

Where $x$ is the bit-score for a match between a profile HMM and a
sequence, $MN$ is the product of the database size and the model length
and, finally $\kappa$ and $\lambda$ are parameters that ensure the
intercept with the y-axis is correct and that the curve matches an
empirical distribution .

In a breakthrough theoretical paper by Sean Eddy \cite{Eddy:2008}, he
showed that the most computationally expensive parameter to estimate
($\lambda$) is a constant i.e.  $\lambda=\ln(2)$.

Thus Equation~\ref{eq:logre} can be rewritten as: 

\begin{eqnarray}
%\begin{align*}
\label{eq:logre2}
\log R.E %&=& \log_{10}\left(E-value_{var}\right) - \log_{10}\left(E-value_{ref}\right)\\
       &=& \log_{10} \left( e^{\lambda x_{var}} \right) - \log_{10} \left( e^{\lambda x_{ref}} \right)  \\
       &=& \left(x_{var}-x_{ref}\right)*\log_{10}(e^{\lambda})                                   \nonumber \\
       &=& -DBS*constant \nonumber 
%\end{align*}
\end{eqnarray}

If the base for the exponential and the logarithms had been equal,
then $constant$ the constant would equal $\lambda$. In either case, a
constant multiplied by the difference between two bitscores is all
that remains.

\subsection{FATHMM}

Shihab \emph{et al} (2013) define the following unweighted measure for
estimating the significance of a single non-synonymous SNP (the
weighted version is trained to discriminate human disease from
polymorphic variation, therefore is not directly comparable to our
general approach) \cite{Shihab:2013}. Their metric is a logit or log-odds value, comparing the
emission probability of the wild-type variant ($P_{ref}$) and a mutant
variant ($P_{var}$) when the mutant is a single, non-synonymous point
mutation (i.e. not multiple point mutations or indels):

\begin{eqnarray}
%\begin{align*}
\label{eq:fathmm}
unweighted &=& \ln \left( \frac{P_{var}/(1-P_{var})}{P_{ref}/(1-P_{ref})} \right)\\
\label{eq:fathmm1}
           &=& \ln \left( \frac{P_{var}}{P_{ref}} \right) + \ln \left( \frac{1-P_{ref}}{1-P_{var}} \right) \\ %\nonumber \\
\label{eq:fathmm2}
           &\approx& -DBS + \ln \left( \frac{1-P_{ref}}{1-P_{var}} \right)                                
%\end{align*}
\end{eqnarray}

The value $1-P_{ref}$ and  $1-P_{var}$ can be re-written as the following summation: 

\begin{eqnarray} 
\label{eq:logitw}
1-P_{ref}=\sum_{i\in{amino-acids},i\ne ref} P_i\\
\label{eq:logitm}
1-P_{var}=\sum_{j\in{amino-acids},j\ne var} P_j
\end{eqnarray}

Equations~\ref{eq:logitw}\&\ref{eq:logitm} share 18 terms (for each of
the 20 amino acids, less the ones corresponding to the wild-type (w)
and mutant (m) variants. Therefore, $1-P_{ref} \approx 1-P_{var}$ for most
realistic biological results. As a consequence, the second term of
Equation~\ref{eq:fathmm2} is approximately zero (or at least, modest
in comparison to the first term when there is a large difference
between $P_{ref}$ and $P_{var}$). Therefore a difference between bitscores is
the term that dominates Equation~\ref{eq:fathmm}.% (see the discussion below). 

\subsection{HMMvar}

Liu \emph{et al} (2014\&2015) present a similar metric to \emph{FATHMM}
for estimating the significance of variation using a profile-HMM based
approach, with some important differences in the implementation
\cite{Liu:2014,Liu:2015}.  The authors calculate the probability of
each sequence directly from \emph{HMMER3} bitscores using $P =
P_{null} *e^B$, where $P_{null} = exp(l*\log(P_1)+\log(1-P_1))$, the
sequence length is $l$ and $P_1 = \frac{350}{351}$.

\begin{eqnarray}
%\begin{align*}
\label{eq:hmmvar}
S &=&  \frac{P_{ref}/(1-P_{ref})}{P_{var}/(1-P_{var})} \\
\log_2\left(S\right) &=&  \log_2\left(\frac{P_{ref}/(1-P_{ref})}{P_{var}/(1-P_{var})}\right) \\
                     &=&  \log_2\left(\frac{P_{ref}}{P_{var}}\right) - \\ 
                     & &  \log_2\left(\frac{1-P_{var}}{1-P_{ref}}\right) \\
                     &\approx& -\log_2\left(\frac{1-P_{var}}{1-P_{ref}}\right) - DBS
%\end{align*}
\end{eqnarray}

Liu \emph{et al} do discuss the possibility of using $DBS$, which they
call $D$, they conclude that ``prediction results using $D$ were not
better than for $S$, and hence are not reported here''.

\section{Discussion}

As a consequence, the measures used by the \emph{$\log$R.E-value}
(Equation~\ref{eq:logre2}) and the \emph{FATHMM}
(Equation~\ref{eq:fathmm2}) approach are approximations to the more
direct estimation of significance, $DBS$. In the case of
\emph{FATHMM}, only single point mutations are considered, missing the
wealth of variation due to insertions, deletions, multiple SNPs and
and other larger-scale variants.

The \emph{HMMvar} approach is similar to DBS, this method employs full
length HMMs (including indel states), just like $DBS$. 

Consequently, $DBS$ is a direct measure of the potential impact of
genetic variation, that can be used on small as well as large and
complex variants. We propose that this metric can be used to evaluate
both population variation as well as variation between species. The
mean of the distribution should be approximately zero, while the
variance will increase with increasing phylogenetic distance (and
different levels of selection). 

One factor that may have an undue influence on $DBS$ is in the rare
cases where the optimal alignment between a the profile and the
variant and the profile and the canonical sequence differ. For
example, \emph{HMMER3} currently only has a local mode (i.e. no
``glocal'' option). As a result, split matches can occur, and
alignment slippage is also possible, particularly for repetitive
sequences.

One way to mitigate these possibilities is to use \emph{Forward
  Scores}, which rather than reporting just the value for an optimal
alignment, reports instead the sum of all possible alignments between
a query sequence and the profile model. 

%----------------------------------------------------------------------------------------
%	REFERENCE LIST
%----------------------------------------------------------------------------------------
\bibliographystyle{unsrt}
\bibliography{paulall}

\end{document}


